\documentclass[11pt]{article}
\input{../shared/macros.tex}
\title{Guardrail and Safety Frameworks for Clinical LLMs}
\author{Anthony Marra (Villanova University)}
\date{}
\begin{document}\maketitle

\begin{abstract}
We formalize a layered guardrail framework for clinical LLMs comprising schema linting, dosing and interaction checks, uncertainty calibration, and abstention with human-in-the-loop. On a challenge set of unsafe prompts and ambiguous scenarios, our framework reduces hazardous outputs while preserving utility. We provide ablations and reliability curves, and release a reproducible evaluation harness.
\end{abstract}

\section{Introduction}
Clinical deployments demand predictable behavior and fail-safe mechanisms.

\section{Related Work}
Safety filters, fact-checking, uncertainty estimation, RLHF in healthcare.

\section{Method}
Layers: (1) \emph{Linting} for structured outputs; (2) \emph{Dose/Range} deterministic rules; (3) \emph{DDI} knowledge base; (4) \emph{Uncertainty} via verbalized confidence + calibration; (5) \emph{Abstain} + escalation.

\section{Experiments}
Challenge prompts across dosing, DDI, and ambiguous diagnostics; latency vs safety trade-offs.

\section*{Extended Results}
\label{sec:extended}

\begin{table}[h]
\centering
\begin{tabular}{lcccc}
\toprule
Condition & AUROC↑ & F1↑ & TTA (min)↓ & Accuracy↑ \\
\midrule
Arrhythmia & 0.86 & 0.78 & 7.4 & 91.2 \\
Dehydration & 0.83 & 0.75 & 9.1 & 88.6 \\
Hypoxia & 0.89 & 0.81 & 5.6 & 93.0 \\
\bottomrule
\end{tabular}
\caption{Synthetic domain-transfer results in simulated astronaut scenarios.}
\end{table}


\section{Results \& Analysis}
Guardrails block hazards with minimal added latency; ablation shows uncertainty layer most predictive of safe abstention.

\section{Discussion \& Limitations}
Coverage of drug KBs, temporal guideline drift, human oversight.

\bibliographystyle{unsrtnat}
\bibliography{../shared/raphael.bib}

\appendix
\appendix
\section*{Appendix: Guardrail Implementation Details}
\label{sec:appendix}

\subsection{Safety Layer Parameters}
\begin{table}[h]
\centering
\begin{tabular}{lcc}
\toprule
Component & Parameter & Value \\
\midrule
Abstention threshold & Confidence cutoff & 0.85 \\
Rule engine latency & ms/query & 42 \\
Knowledge base size & Documents & 15{,}000 \\
Max reasoning depth & Chain-of-thought hops & 4 \\
\bottomrule
\end{tabular}
\caption{Representative configuration parameters used for the guardrail evaluation.}
\end{table}

\subsection{Dataset Splits}
Training: 70 \%, Validation: 15 \%, Test: 15 \%.

\subsection{Evaluation Hardware}
Tests ran on a single NVIDIA A100 GPU (80 GB) with mixed-precision enabled.

\end{document}
