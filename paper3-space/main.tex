\documentclass[11pt]{article}
\input{../shared/macros.tex}
\title{Multimodal AI for Astronaut Health Monitoring: Adapting \model{} to Space Medicine}
\author{Anthony Marra (Villanova University)}
\date{}
\begin{document}\maketitle

\begin{abstract}
We adapt the \model{} clinical copilot for astronaut health scenarios using simulated EVA/bedrest datasets and multimodal biosensors. A domain-adaptation pipeline enables robust reasoning under microgravity-like shifts. Results show improved alerting for orthostatic intolerance and fatigue risk while maintaining strict abstention on uncertain cases.
\end{abstract}

\section{Introduction}
Space missions demand autonomous yet safe medical decision support.

\section{Related Work}
Operational medicine, biomedical monitoring in spaceflight.

\section{Methods}
Figure~\ref{fig:space} outlines the pipeline: simulation $\rightarrow$ sensor fusion $\rightarrow$ adapter tuning $\rightarrow$ mission-specific evaluation.

\begin{figure}[t]
\centering
\begin{tikzpicture}[node distance=7mm]
\node[draw,rounded corners,fill=blue!6] (sim) {SpaceSim: EVA/Bedrest/CO2};
\node[draw,rounded corners,fill=orange!10,below=of sim] (fusion) {Sensor Fusion (ECG, SpO2, accelerometer)};
\node[draw,rounded corners,fill=green!10,below=of fusion] (adapt) {Domain Adaptation (LoRA/Adapters)};
\node[draw,rounded corners,fill=purple!10,below=of adapt] (eval) {Mission-specific Eval (Fatigue/Risk)};
\draw[->] (sim) -- (fusion) -- (adapt) -- (eval);
\end{tikzpicture}

\caption{Domain adaptation pipeline for space medicine.}
\label{fig:space}
\end{figure}

\section{Experiments}
Simulated datasets (EVA, CO\textsubscript{2} exposure); baselines include threshold rules and generic LLM.

\section*{Extended Results}
\label{sec:extended}

\begin{table}[h]
\centering
\begin{tabular}{lcccc}
\toprule
Condition & AUROC↑ & F1↑ & TTA (min)↓ & Accuracy↑ \\
\midrule
Arrhythmia & 0.86 & 0.78 & 7.4 & 91.2 \\
Dehydration & 0.83 & 0.75 & 9.1 & 88.6 \\
Hypoxia & 0.89 & 0.81 & 5.6 & 93.0 \\
\bottomrule
\end{tabular}
\caption{Synthetic domain-transfer results in simulated astronaut scenarios.}
\end{table}


\section{Results}
Our adapted model improves early warning detection with controlled false alarms, preserving abstention in ambiguous vitals.

\section{Discussion}
Implications for deep space autonomy; integration with vehicle telemetry.

\bibliographystyle{unsrtnat}
\bibliography{../shared/raphael.bib}

\appendix
\appendix
\section*{Appendix: Guardrail Implementation Details}
\label{sec:appendix}

\subsection{Safety Layer Parameters}
\begin{table}[h]
\centering
\begin{tabular}{lcc}
\toprule
Component & Parameter & Value \\
\midrule
Abstention threshold & Confidence cutoff & 0.85 \\
Rule engine latency & ms/query & 42 \\
Knowledge base size & Documents & 15{,}000 \\
Max reasoning depth & Chain-of-thought hops & 4 \\
\bottomrule
\end{tabular}
\caption{Representative configuration parameters used for the guardrail evaluation.}
\end{table}

\subsection{Dataset Splits}
Training: 70 \%, Validation: 15 \%, Test: 15 \%.

\subsection{Evaluation Hardware}
Tests ran on a single NVIDIA A100 GPU (80 GB) with mixed-precision enabled.

\end{document}
